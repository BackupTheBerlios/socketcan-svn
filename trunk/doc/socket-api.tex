% $Id: socket-api.tex,v 2.0 2006/04/13 10:37:19 ethuerm Exp $

\newpage
\section{Allgemeine Hinweise zur Socket-API beim \LL}
\label{socket-api}

F�r die Kommunikation auf dem CAN-Bus wird eine neue Protokoll-Familie
\verb+PF_CAN+ im Socket-Layer implementiert. Aus Anwender-
bzw. Programmierersicht wird mit den \LL-Sockets analog zu
Internet-Protokoll-Sockets (Protokoll-Familie \verb+PF_INET+) mit den
�blichen Systemaufrufen \man{socket}{2}, \man{bind}{2}, \man{listen}{2},
\man{accept}{2}, \man{connect}{2}, \man{read}{2},
\man{write}{2} und \man{close}{2} genutzt. Siehe dazu auch die
Einleitung in Kapitel \ref{intro}.\\

Im Gegensatz zur Adressstruktur der Internet-Adressen
(\verb+sockaddr_in+) ben�tigt die Adressierung der PF\_CAN-Sockets
andere Inhalte. Die Adressstruktur \verb+sockaddr_can+ ist in der
Include-Datei \verb+af_can.h+ definiert:

\begin{code}
struct sockaddr_can {
    sa_family_t   can_family;
    int           can_ifindex;
    union {
        struct { canid_t rx_id, tx_id; } tp16;
        struct { canid_t rx_id, tx_id; } tp20;
        struct { canid_t rx_id, tx_id; } mcnet;
    } can_addr;
};
\end{code}

Neben dem Interface-Index des CAN-Interfaces sind hierbei besonders
f�r die jeweiligen Transport-Protokolle die CAN-IDs \verb+tx_id+ und
\verb+rx_id+ relevant. Transport-Protokolle bilden auf dem CAN-Bus
auf zwei CAN-IDs eine virtuelle Punkt-zu-Punkt-Verbindung ab.\\

Eine weitere wichtige Struktur stellt das CAN-Frame dar, dass in
der Include-Datei \verb+can.h+ definiert ist:

\begin{code}
typedef __u32 canid_t;

struct can_frame {
    canid_t can_id;      /* 32 bit CAN_ID + EFF/RTR flags */
    __u8    can_dlc;     /* data length code: 0 .. 8 */
    __u8    data[8] __attribute__ ((aligned(8)));
};

/* special address description flags for the CAN_ID */
#define CAN_EFF_FLAG 0x80000000U /* EFF/SFF is set in the MSB */
#define CAN_RTR_FLAG 0x40000000U /* remote transmission request */

/* valid bits in CAN ID for frame formats */
#define CAN_SFF_MASK 0x000007FFU /* standard frame format (SFF) */
#define CAN_EFF_MASK 0x1FFFFFFFU /* extended frame format (EFF) */
\end{code}

Die definierte Struktur \verb+can_frame+ enth�lt die Elemente eines
CAN-Frame, wie es auf dem CAN-Bus definiert ist. Die Anordnung der
Nutzdaten (Byte-Order, Word-Order, Little/Big Endian) ist auf dem
CAN-Bus generell nicht definiert, weshalb die Datenelemente
\verb+data[]+ als Array von 8 Byte ausgef�hrt sind. Da die
Datenelemente allerdings auf einer 8 Byte Speichergrenze ausgerichtet,
ist hier auch ein Zugriff bis zu einer Breite von 64 Bit m�glich:

\begin{code}
#define U64_DATA(p) (*(unsigned long long*)(p)->data)

U64_DATA(&myframe) = 0xFFFFFFFFFFFFFFFFULL;
U64_DATA(&myframe) = 0;
\end{code}

Durch die Trennung der Informationen in die Include-Dateien
\verb+af_can.h+ und \verb+can.h+ ist zum Erstellen eines
CAN-Netzwerk-Treibers nur die Include-Datei \verb+can.h+ n�tig.\\

\subsection{Zeitstempel}

F�r die Anwendungen im CAN-Umfeld ist h�ufig ein genauer Zeitstempel
von Interesse, der den Empfangszeitpunkt einer Nachricht vom CAN-Bus
wiedergibt. Ein solcher zugeh�riger Zeitstempel kann �ber ein
\man{ioctl}{2} nach dem Lesen einer Nachricht vom Socket ausgelesen
werden. Dieses gilt auch f�r die Sockets von Transportprotokollen,
wobei hier der Zeitstempel der letzten zugeh�rigen TPDU ausgegeben
wird. Der Aufruf - z.B. \verb+ioctl(s, SIOCGSTAMP, &tv)+ - wird in den
jeweiligen Testprogrammen zur Veranschaulichung verwendet.\\

Die Zeitstempel haben unter Linux eine Aufl�sung von einer Mikrosekunde
und werden beim Empfang eines CAN-Frame im CAN-Networkdevice
automatisch gesetzt.

