% $Id$

\newpage
\section{RAW-Sockets}
\label{rawsocket}

RAW-Sockets erlauben, Nachrichten direkt auf einem CAN-Bus zu
senden und alle Nachrichten, die auf einem CAN-Bus �bertragen
werden, zu lesen. Ge�ffnet wird ein RAW-Socket durch

\begin{code}
s = socket(PF_CAN, SOCK_RAW, 0);
\end{code}

Der ge�ffnete Socket muss zun�chst mittels \man{bind}{2} an einen CAN-Bus
gebunden werden.  Dabei spielen die f�r Transportprotokolle ben�tigten
Adress-Elemente \verb|tx_id| und \verb|rx_id| in der Struktur
\verb|struct sockaddr_can| keine Rolle. Der 
folgende Code bindet den ge�ffneten Socket s an das CAN-Interface
   \verb|can1|:

\begin{code}
struct sockaddr_can addr;
struct ifreq ifr;

addr.can_family = AF_CAN;
strcpy(ifr.ifr_name, "can1");
ioctl(s, SIOCGIFINDEX, &ifr);
addr.can_ifindex = ifr.ifr_ifindex;
bind(s, (struct sockaddr *)&addr, sizeof(addr));
\end{code}

Es k�nnen nun mit \man{read}{2} alle auf dem Bus empfangenen CAN-Frames
gelesen und mit \man{write}{2} beliebige CAN-Frames gesendet werden.\\

Die mit \man{read}{2} und \man{write}{2} �bertragenen Daten haben die
Struktur \verb|struct can_frame|. Jeder zu sendende CAN-Frame muss 
mit \emph{einem} Aufruf von \man{write}{2} �bergeben werden und
empfangene CAN-Frame m�ssen mit \emph{einem} Aufruf von \man{read}{2}
gelesen werden.\\

Zum gleichzeitigen Empfang von Nachrichten \emph{aller}
CAN-Netzwerk-Interfaces (z.B. mit 
\man{recvfrom}{2}) ist als Interface-Index Null (im Beispiel:
\verb+addr.can_ifindex = 0;+) einzutragen. Das Senden von CAN-Frames
�ber einen solchen RAW-Socket muss dann �ber \man{sendmsg}{2} erfolgen.\\

Der RAW-Socket bietet eine einfache Filterfunktion mit der Bereiche
von CAN-IDs aus dem Datenstrom ausgefiltert werden k�nnen. Dazu kann
noch vor dem Aufruf von \man{bind}{2} mit dem Systemaufruf
\man{setsockopt}{2} ein Array einfacher Filter gesetzt werden. In diesem
Beispiel sollen alle CAN-IDs von 0x200 - 0x2FF durchgelassen werden:

\begin{code}
struct can_filter rfilter;

rfilter.can_id   = 0x200; /* SFF frame */
rfilter.can_mask = 0xF00;

setsockopt(s, SOL_CAN_RAW, CAN_RAW_FILTER, &rfilter, sizeof(rfilter));
\end{code}

Der Filter l��t sich mit \verb+rfilter.can_id |= CAN_INV_FILTER;+ auch
invertieren, wodurch in diesem Fall die CAN-IDs von 0x200 - 0x2FF
nicht durchgelassen werden w�rden.\\

\begin{code}
struct can_filter rfilter[4];

rfilter[0].can_id   = 0x80001234;   /* Exactly this EFF frame */
rfilter[0].can_mask = CAN_EFF_MASK; /* 0x1FFFFFFFU all bits valid */
rfilter[1].can_id   = 0x123;        /* Exactly this SFF frame */
rfilter[1].can_mask = CAN_SFF_MASK; /* 0x7FFU all bits valid */
rfilter[2].can_id   = 0x200 | CAN_INV_FILTER; /* all, but 0x200-0x2FF */
rfilter[2].can_mask = 0xF00;        /* (CAN_INV_FILTER set in can_id) */
rfilter[3].can_id   = 0;            /* don't care */
rfilter[3].can_mask = 0;            /* ALL frames will pass this filter */

setsockopt(s, SOL_CAN_RAW, CAN_RAW_FILTER, &rfilter, sizeof(rfilter));
\end{code}

F�r die Anwendung der Filterfunktion muss die Include-Datei \verb+raw.h+
eingebunden werden. Eine Ver�nderung des Filters zur Laufzeit ist �ber
weitere Aufrufe von \man{setsockopt}{2} m�glich.

\newpage
\subsection{Testprogramme}
\begin{description}
\item[candump] ist ein Programm, das �ber den RAW-Socket CAN-Frames
von einem oder mehreren CAN-Device(s) einliest und in lesbarer Form -
 auf Wunsch mit Zeitstempeln - ausgibt. Die beschriebene
 Filterfunktion der RAW-Sockets (s.o.) ist �ber Komandozeilenparameter
 einstellbar. Ebenso wie eine Ausgabe im bekannten ASC-Format.

Wird \verb+candump+ ohne Parameter aufgerufen, erscheint ein Hilfetext.

\begin{code}
hartko@pplinux1:~/llcf/src/test > ./candump 
Usage: candump [can-interfaces]
Options: -m <mask>   (default 0x00000000)
         -v <value>  (default 0x00000000)
         -i <0|1>    (inv_filter)
         -t <type>   (timestamp: Absolute/Delta/Zero)
         -c          (color mode)
         -s <level>  (silent mode - 1: animation 2: nothing)
         -b <can>    (bridge mode - send received frames to <can>)
         -a          (create ASC compatible output)
         -1          (increment interface numbering in ASC mode)
         -A          (enable ASCII output)

When using more than one CAN interface the options
m/v/i have comma seperated values e.g. '-m 0,7FF,0'
\end{code}

{\large Beispiele f�r die Benutzung von {\tt candump}:}\\
\\
Einfaches Anzeigen von zwei CAN-Bussen. Timestamps beginnen bei Null.

\begin{code}
hartko@pplinux1:~/llcf/src/test > candump can0 can1 -t z
(0.000000) can0  3FC  [1] 05 
(0.001185) can0   64  [8] 20 14 3F 16 B8 0B 98 3A 
(0.002396) can0   66  [8] 39 00 A1 45 00 00 00 00 
(0.015395) can0   C9  [6] 13 01 00 00 10 27 
(0.028665) can1  110  [3] 87 00 00 
(0.049990) can0  3FC  [1] 05 
(0.051176) can0   64  [8] 20 14 3F 16 B8 0B 98 3A 
(0.052386) can0   66  [8] 39 00 A1 45 00 00 00 00 
(0.065397) can0   C9  [6] 13 01 00 00 10 27 
(0.099974) can0  3FC  [1] 05 
(0.101159) can0   64  [8] 20 14 3F 16 B8 0B 98 3A 
\end{code}

Einfaches Anzeigen von zwei CAN-Bussen. Timestamps beginnen bei
Null. Ausgabe im ASC-Format. Die Kanalnummern werden um 1 erh�ht
(\verb+can0+ $\Rightarrow$ 1, \verb+can1+ $\Rightarrow$ 2).

\begin{code}
hartko@pplinux1:~/llcf/src/test > candump can0 can1 -t z -a -1
date Tue Oct 18 09:39:57 2005
base hex  timestamps absolute
no internal events logged
   0.0000 1  3FC             Rx   d 1 05 
   0.0011 1  64              Rx   d 8 20 14 3F 14 B8 0B 98 3A 
   0.0023 1  66              Rx   d 8 49 00 A1 45 00 00 00 00 
   0.0154 1  C9              Rx   d 6 13 01 00 00 10 27 
   0.0286 2  110             Rx   d 3 87 00 00 
   0.0500 1  3FC             Rx   d 1 05 
   0.0512 1  64              Rx   d 8 20 14 3F 14 B8 0B 98 3A 
   0.0524 1  66              Rx   d 8 49 00 A1 45 00 00 00 00 
   0.0654 1  C9              Rx   d 6 13 01 00 00 10 27 
   0.0999 1  3FC             Rx   d 1 05 
   0.1011 1  64              Rx   d 8 20 14 3F 14 B8 0B 98 3A 
\end{code}

Filtern von Nachrichten auf \verb+can0+ mit Bit-Maske 0x7FC und
Vergleichswert 0x66. Zeitstempel mit Differenzzeit.

\begin{code}
hartko@pplinux1:~/llcf/src/test > candump can0 -m 0x7FC -v 0x66 -t d 
CAN ID filter[0] for can0 set to mask = 000007FC, value = 00000066 
(0.000000) can0   64  [8] 1B 14 3F 21 B8 0B 98 3A 
(0.001202) can0   66  [8] B9 DA A0 45 00 00 00 00 
(0.048833) can0   64  [8] 1B 14 3F 21 B8 0B 98 3A 
(0.001192) can0   66  [8] EB DA A0 45 00 00 00 00 
(0.048790) can0   64  [8] 1B 14 3F 21 B8 0B 98 3A 
\end{code}

Filtern von Nachrichten auf \verb+can0+ mit Bit-Maske 0x7FC und
Vergleichswert 0x66. Der CAN-Bus \verb+can1+ wird ohne Filterung angegeigt.

\begin{code}
hartko@pplinux1:~/llcf/src/test > candump can0 can1 -m 0x7FC,0 -v 0x66,0 -t d 
CAN ID filter[0] for can0 set to mask = 000007FC, value = 00000066 
(0.000000) can0   64  [8] 20 14 3F 14 B8 0B 98 3A 
(0.001202) can0   66  [8] 48 00 A1 45 00 00 00 00 
(0.048794) can0   64  [8] 20 14 3F 14 B8 0B 98 3A 
(0.001201) can0   66  [8] 48 00 A1 45 00 00 00 00 
(0.026214) can1  110  [3] 87 00 00 
(0.003006) can1  41B  [4] 1C 12 02 FF 
(0.019612) can0   64  [8] 20 14 3F 14 B8 0B 98 3A 
(0.001201) can0   66  [8] 48 00 A1 45 00 00 00 00 
(0.048770) can0   64  [8] 20 14 3F 14 B8 0B 98 3A 
\end{code}

Invertiertes Filtern von Nachrichten auf \verb+can0+ mit Bit-Maske 0x7FC und
Vergleichswert 0x66. Der CAN-Bus \verb+can1+ wird ohne Filterung
angegeigt. Der Timestamp wird absolut ausgegeben, d.h. in Sekunden
seit 01.01.1970. Analog \verb-'date +%s'- - siehe \man{date}{1}. 

\begin{code}
hartko@pplinux1:~/llcf/src/test > candump can0 can1 -m 0x7FC,0 -v 0x66,0 -i 1,0 -t a 
CAN ID filter[0] for can0 set to mask = 000007FC, value = 00000066 (inv_filter)
(1129625880.726372) can0   C9  [6] 13 01 00 00 10 27 
(1129625880.739543) can1  110  [3] 87 00 00 
(1129625880.760949) can0  3FC  [1] 05 
(1129625880.776377) can0   C9  [6] 13 01 00 00 10 27 
(1129625880.810983) can0  3FC  [1] 05 
(1129625880.811580) can1  41C  [4] 1D 12 02 FF 
(1129625880.826379) can0   C9  [6] 13 01 00 00 10 27 
(1129625880.839544) can1  110  [3] 87 00 00 
(1129625880.860955) can0  3FC  [1] 05 
(1129625880.876380) can0   C9  [6] 13 01 00 00 10 27 
(1129625880.910986) can0  3FC  [1] 05 
\end{code}

\item[tst-raw] ist ein Programm zum Testen des RAW-Sockets. Es nutzt
den \man{read}{2} Systemcall. 
\item[tst-raw-filter] ist ein Programm zum Testen der Filterfunktion
des RAW-Sockets. Es nutzt den \man{recvfrom}{2} Systemcall.
\item[tst-raw-sendto] ist ein Programm zum Senden eines CAN-Frame auf
einem nicht an ein besonders Interface gebundenen RAW-Socket mit
dem \man{sendto}{2} Systemcall.
\item[canecho] ist ein Programm, dass nach dem Empfang eines
CAN-Frames dieses mit einer um 1 erh�hten CAN-ID wieder aussendet. Es
war f�r die ersten Versuche mit einem realen CAN-Bus implementiert
worden. Seit dem \LL\ V0.6 ist jedoch die lokale Echofunktionalit�t
realisiert, so dass \verb+canecho+ nur noch dazu geeignet ist, einen
Volllast-Test auszuf�hren ...
\end{description}
