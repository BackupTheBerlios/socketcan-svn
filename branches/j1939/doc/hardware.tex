% $Id$

\newpage
\section{Unterst�tzte CAN-Hardware}
\label{hardware}

Bisherige Realisierungen von CAN-Treibern unter
Linux und auch anderen Betriebssystemen sind nach dem
Zeichen-Treibermodell (dem so genannten Character-Device) ausgef�hrt.\\

Im Unterschied dazu setzt das \LLCF\ auf
CAN-Treiber nach dem Netzwerk-Treibermodell auf (so genannte
Network-Devices), die es erm�glichen, dass mehrere Anwendungen
gleichzeitig auf einem CAN-Bus arbeiten k�nnen.\\

Wenngleich ein Treiber nach dem Netzwerk-Treibermodell einfacher zu
realisieren ist, sind die bei einer kommerziellen CAN-Hardware
beigelegten Treiber f�r das \LL\ so nicht einsetzbar. Ist der
Quellcode des beigelegten Treibers verf�gbar, kann man diesen
allerdings so modifizieren, dass er sich nicht als Character-Device
sondern als Network-Device im Linux-Kernel registriert und entsprechend
andere Schnittstellen des Kernel bedient.\\

Der unter \verb+src/drivers/sja1000+ realisierte
Philips-SJA1000-Treiber ist eine 
komplette Neuentwicklung und kann als Beispiel f�r einen
CAN-Netzwerktreiber genommen werden. Derzeit unterst�tzt das \LL\
ausschlie�lich passive CAN-Karten, weil dieses f�r den Linux-Kernel im
Gegensatz zu anderen Betriebssystemen problemlos m�glich ist. Ein
modifizierter Treiber f�r eine aktive PCMCIA-Karte ist in Arbeit.\\

Derzeit werden folgende CAN-Hardware-Komponenten unterst�tzt:
\subsection{PC104 / ISA / plain access}

In diesen Karten liegen die SJA1000-Controller linear im Adressraum.\\
Z.B. http://www.peak-system.com/db/de/pcanpc104.html\\

M�gliche Treiber:
\begin{itemize}
\item \LL-SJA1000-Treiber in \verb+src/drivers/sja1000+ (empfohlen)
\item Modifizierter Linux-Treiber v2.15 von PEAK-System (auf Anfrage)
\end{itemize}

\subsection{PCI}

In diesen Karten liegen die SJA1000-Controller linear im
PCI-Adressraum.\\
Z.B. http://www.peak-system.com/db/de/pcanpci.html\\

M�gliche Treiber:
\begin{itemize}
\item Modifizierter Linux-Treiber v2.15 von PEAK-System (auf Anfrage)
\end{itemize}

\subsection{Parallelport}

Ben�tigt Linux-Parport-Unterst�tzung.\\
Z.B. http://www.peak-system.com/db/de/pcandongle.html\\

M�gliche Treiber:
\begin{itemize}
\item Modifizierter Linux-Treiber v2.15 von PEAK-System (auf Anfrage)
\end{itemize}

\subsection{USB}

USB-CAN-Adapter.\\
http://www.peak-system.com/db/de/pcanusb.html\\

M�gliche Treiber:
\begin{itemize}
\item Modifizierter Linux-Treiber v2.15 von PEAK-System
\end{itemize}

\subsection{PCMCIA}

Passive PCMCIA-Karte mit zwei SJA1000-Controllern.\\
http://www.ems-wuensche.com/catalog/english/datasheet/htm/cpccard\_e.htm\\

M�gliche Treiber:
\begin{itemize}
\item Modifizierter Linux-Treiber cdkl-1.12 von EMS W�nsche (auf Anfrage)
\end{itemize}

Aktive PCMCIA-Karte mit zwei SJA1000-Controllern.\\
http://www.kvaser.com/prod/hardware/lapcan\_i.htm\\
http://www.kvaser.com/prod/hardware/lapcan\_ii.htm\\

M�gliche Treiber (in Arbeit!):
\begin{itemize}
\item Modifizierter Linux-Treiber v4.1beta von Kvaser
\end{itemize}


\subsection{Virtual CAN Bus (vcan)}

Der virtuelle CAN-Bus-Treiber realisiert ein logisches
CAN-Network-Device, �ber das Anwendungen auf einem System ohne real
vorhandene CAN-Hardware kommunizieren k�nnen. Die Idee entspricht
einem Loopback-Device, wobei die Loopback-Funktionalit�t (siehe
Kapitel \ref{intro}) bereits im \LL-Rahmen realisiert ist. Der
vcan-Treiber ist Bestandteil des \LLCF.
