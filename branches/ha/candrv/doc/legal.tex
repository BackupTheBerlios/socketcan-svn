% $Id$

\newpage
\section{Rechtliche Hinweise}
\label{legal}

Volkswagen geht davon aus, dass diese rechtlichen Hinweise vom
Anwender gelesen, verstanden und akzeptiert worden sind.

\medskip
Im Quellcode des \LLCF\ findet man folgenden Hinweis:

\begin{code}
/*
 * Copyright (c) 2002-2005 Volkswagen Group Electronic Research
 * All rights reserved.
 *
 * Redistribution and use in source and binary forms, with or without
 * modification, are permitted provided that the following conditions
 * are met:
 * 1. Redistributions of source code must retain the above copyright
 *    notice, this list of conditions, the following disclaimer and
 *    the referenced file 'COPYING'.
 * 2. Redistributions in binary form must reproduce the above copyright
 *    notice, this list of conditions and the following disclaimer in the
 *    documentation and/or other materials provided with the distribution.
 * 3. Neither the name of Volkswagen nor the names of its contributors
 *    may be used to endorse or promote products derived from this software
 *    without specific prior written permission.
 *
 * Alternatively, provided that this notice is retained in full, this
 * software may be distributed under the terms of the GNU General
 * Public License ("GPL") version 2 as distributed in the 'COPYING'
 * file from the main directory of the linux kernel source.
 *
 * The provided data structures and external interfaces from this code
 * are not restricted to be used by modules with a GPL compatible license.
 *
 * THIS SOFTWARE IS PROVIDED BY THE COPYRIGHT HOLDERS AND CONTRIBUTORS
 * "AS IS" AND ANY EXPRESS OR IMPLIED WARRANTIES, INCLUDING, BUT NOT
 * LIMITED TO, THE IMPLIED WARRANTIES OF MERCHANTABILITY AND FITNESS FOR
 * A PARTICULAR PURPOSE ARE DISCLAIMED. IN NO EVENT SHALL THE COPYRIGHT
 * OWNER OR CONTRIBUTORS BE LIABLE FOR ANY DIRECT, INDIRECT, INCIDENTAL,
 * SPECIAL, EXEMPLARY, OR CONSEQUENTIAL DAMAGES (INCLUDING, BUT NOT
 * LIMITED TO, PROCUREMENT OF SUBSTITUTE GOODS OR SERVICES; LOSS OF USE,
 * DATA, OR PROFITS; OR BUSINESS INTERRUPTION) HOWEVER CAUSED AND ON ANY
 * THEORY OF LIABILITY, WHETHER IN CONTRACT, STRICT LIABILITY, OR TORT
 * (INCLUDING NEGLIGENCE OR OTHERWISE) ARISING IN ANY WAY OUT OF THE USE
 * OF THIS SOFTWARE, EVEN IF ADVISED OF THE POSSIBILITY OF SUCH
 * DAMAGE.
 *
 * Send feedback to <socketcan-users@lists.berlios.de>
 *
 */
\end{code}


\subsection{Erweiterter Haftungsausschluss}

Im Geltungsbereich der deutschen Rechtsprechung besteht auch bei der
kostenlosen �berlassung bei grob fahrl�ssigen oder 
vors�tzlich verschwiegenen M�ngeln die M�glichkeit, den Urheber f�r
entstandene (Folge-)Sch�den haftbar machen zu k�nnen.\\ 

Wenngleich die Autoren bem�ht sind, eine fehlerfreie Software zur
 Verf�gung zu stellen, lassen sich Fehler nicht generell
 ausschlie�en. Aus
 diesem Grunde erkl�rt sich der Anwender mit dem Einsatz des  
\LLCF\ damit einverstanden, den Haftungsausschluss gegen�ber den
Autoren und der Volkswagen AG {\bf uneingeschr�nkt} anzuerkennen und
auf jedwede rechtlichen M�glichkeiten/Forderungen beim Auftreten von
M�ngeln/Sch�den/Folgesch�den durch den Einsatz des \LL\ zu
verzichten.

\subsection{Logo}

Das Logo des \LLCF\ ( '/dev/$<$ beetle $>$' ) ist als zusammengesetzte
Bildmarke Eigentum der Volkswagen AG. Es symbolisiert die Integration
des Fahrzeugs in eine Umgebung der Standard-Informationstechnologie.

\subsection{Linux Module License}

Teile der vollst�ndigen \LL-Distribution unterliegen nicht der GNU Public
License sondern sind propriet�rer Code der Volkswagen
AG. Richtigerweise ist dieses auch im Quellcode der einzelnen
Kernelmodule mit dem Makro {\mbox{\bf MODULE\_LICENSE(\grqq Volkswagen
Group closed source\grqq)}} markiert. Beim Laden der \LL-Module in den
Kernel kann daher beispielsweise folgende Fehlermeldung auftreten:

\begin{code}
Warning: loading can-tp20.o will taint the kernel: Volkswagen Group closed source
See http://www.tux.org/lkml/#export-tainted for information about tainted modules
Module tainted loaded, with warnings
\end{code}

Dieses ist eine Warnung, dass das geladene Modul nicht unter der GNU
Public License steht, was die Funktionsf�higkeit des \LLCF\ jedoch nicht
beeinflusst. Typischerweise werden jedoch Anfragen zu Fehlermeldungen,
die von 'verschmutzten' tainted-Kernels erzeugt wurden, im Allgemeinen
von der Linux-Community nicht kommentiert / beantwortet.\\ 

Module, die unter der GNU Public License stehen, enthalten das Makro
{\mbox{\bf MODULE\_LICENSE(\grqq GPL\grqq)}}.

\subsection{Transportprotokolle}

Die in der vollst�ndigen \LL-Distribution enthaltenen Protokoll-Module f�r die
Transport-Protokolle (derzeit VAG TP1.6, VAG TP2.0, Bosch MCNet)
sind nur in Verbindung mit einem Non Disclosure Agreement (NDA)
f�r Zulieferer der Volkswagen AG erh�ltlich. Diese Protokoll-Module
sind keine Referenz-Treiber. Nach
Auslauf des NDA gelten die im NDA vereinbarten Bedingungen zum
Vernichten von Arbeitsergebnissen/Unterlagen/Quellcode.\\

Der Quelltext f�r die bezeichneten CAN-Transport-Protokolle ist in
klar separierten Modulen und existierte bereits vor einer Integration
in den Linux Kernel:

\begin{code}
This TP code is in clearly seperate modules and had a life outside Linux
from the beginning and does something self-contained that doesn't
really have any impact on the rest of the kernel. The transport
protocol drivers have been originally written for something else and
do not need any but the standard UNIX read/write kind of interfaces.
See <http://www.atnf.csiro.au/people/rgooch/linux/docs/licensing.txt>
\end{code}

MCNet ist ein CAN-Transport-Protokoll der Robert Bosch GmbH. Die
Implementierung ist nach Ma�gabe des MCNet-Disclaimers {\it
ANFORDERUNG DER MCNet-SPEZIFIKATION} vom 13.07.1998
erfolgt. Weitergehende Informationen zu MCNet sind erh�ltlich bei:\\

Robert Bosch GmbH\\
Abteilung K7/EFT62\\
Postfach 77 77 77\\
D-31132 Hildesheim\\

Ansprechpartner:\\
Dr. Uwe Zurm�hl $<$uwe.zurmuehl@de.bosch.com$>$\\
Detlef Rode $<$detlef.rode@de.bosch.com$>$

